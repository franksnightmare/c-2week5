\documentclass[11pt]{article}

\usepackage{times}
\usepackage[english]{babel}

% -----------------------------------------------
% especially use this for you code
% -----------------------------------------------

\usepackage{courier}
\usepackage{listings}
\usepackage{color}
\usepackage{tabularx}
\usepackage{graphicx}

\definecolor{Gray}{gray}{0.95}

\definecolor{mygreen}{rgb}{0,0.6,0}
\definecolor{mygray}{rgb}{0.5,0.5,0.5}
\definecolor{mymauve}{rgb}{0.58,0,0.82}

\lstset{language=C++,
	basicstyle = \normalsize\ttfamily,   % the size and fonts that are used
	tabsize = 2,                    % sets default tabsize
	breaklines = true,              % sets automatic line breaking
	keywordstyle=\color{blue}\ttfamily,
	stringstyle=\color{red}\ttfamily,
	commentstyle=\color{mygreen}\ttfamily,
	numbers=left,
	keepspaces=true,
	showspaces=false,
	showstringspaces=false,
}

\begin{document}

\title{Programming in C/C++ \\
       Exercises set five: STL and GA
}
\date{\today}
\author{Christiaan Steenkist \\
Jaime Betancor Valado \\
Remco Bos \\
}

\maketitle

\section*{Exercise 31, Extracting lines with GAs}
We extracted lines from an input stream by using \texttt{copy}, \texttt{input\_iterator}s and a custom Line class.

There is already an overloaded string extraction operator that we would not be overloading again.
That extraction operator stops extracting at a whitespace which means it does not do what we want it to do.

\subsection*{Code listing}
\lstinputlisting[caption = main.cc]{src/a31/main.cc}

\section*{Exercise 32, You get a promotion!}
We used sort to sort in two cool sorting ways, with promotion!

\subsection*{Code listing}
\lstinputlisting[caption = main.cc]{src/a32/main.cc}

\section*{Exercise 33, Lambda functions}
Lambda functions are used in this program that (currently) counts vowels.

\subsection*{Output}
\lstinputlisting[caption = output]{src/a33/output}

\subsection*{Code listings}
\lstinputlisting[caption = vstring.ih]{src/a33/vstring.ih}
\lstinputlisting[caption = vstring.h]{src/a33/vstring.h}
\lstinputlisting[caption = main.cc]{src/a33/main.cc}
\lstinputlisting[caption = count.cc]{src/a33/count.cc}
\lstinputlisting[caption = countchar.cc]{src/a33/countchar.cc}
\lstinputlisting[caption = vowels.cc]{src/a33/vowels.cc}
\lstinputlisting[caption = vstring.cc]{src/a33/vstring.cc}

\section*{Exercise 34, GA's and removing elements}
Extra items are added if "extra" is found in the input and only the unique items remain.

\subsection*{Inputs}
\lstinputlisting[caption = data]{src/a34/input1}
\lstinputlisting[caption = extra]{src/a34/input2}

\subsection*{Output}
\lstinputlisting[caption = output]{src/a34/output}

\subsection*{Code listing}
\lstinputlisting[caption = main.cc]{src/a34/main.cc}

\subsection*{Exercise 35, Copy and for\_each}
Here we explain the differences between them.

\subsection*{Answers}
The copy generic algorithm copies a series of elements (the range of
the iterator) to an output range (destination).

The for\_each generic algorithm passes a series of elements (the range
of the iterator) as reference to a function that may modify the
series of elements.

\subsection*{Code listings}
\lstinputlisting[caption = copy.cc]{src/a35/copy.cc}
\lstinputlisting[caption = for\_each.cc]{src/a35/for_each.cc}

\end{document}
